\section{Introduction}
% problem, objective, aims, !self-contained document!, challenges, solutions, how effective for the solutions
\subsection{Aims and Objectives}

\par\noindent
The project aimed to study general population protocols \cite{AspnesR2007} and
its two derived model,
network constructor \cite{MS16a} and terminating grid network constructor \cite{Mi17}
(specifically for grid network construction).
It also attempted to experimentally simulate, visualise and compare these protocols
via building the simulator and visualizer.

\subsection{The challenges in the project}
\subsubsection{Heterogeneous for different types of Models}

\par\noindent
The theoretical models involved in three main different models initially originated in
population protocols. These three models share inherently common points but there are also some
conceptual differences in between them. For instance, the network constructor \cite{MS16a} and terminating
grid network constructor \cite{Mi17}
involve state of connections in between two nodes while the original population protocol does not.
The node of terminating grid network constructor has its complexity structurally compared with
the other two types of model.

\subsubsection{Heterogeneous for different types of Protocols}

\par\noindent
The protocols discussed in the related papers \cite{AspnesR2007, MS16a, Mi17} involve
many different protocols. The protocols are totally different on many characteristics,
such as their different computational ability, different ending in either or termination,
computation target. These differences between
protocol to protocol may lead the simulator and visualizer hard to develop and test.

\subsubsection{Human factor: Lacking Experience for model visualization}

\par\noindent
Prior to this project, the author has totally no experiences on model simulation and
also no knowledge on what kind of related library will be involved. Learning may take
more time than its expected.

\subsubsection{Brief introduction to the programme}

\par\noindent
The final programme contains an UI with an fix-sized area to illustrate the interaction process of
a particular protocol and shows the states of elements\footnote{\noindent "Elements" refers nodes in general population protocol,
but also includes edge if the protocol involves edge states.} in the population.
In addition, it contains a information panel containing some related
information with regard to the population itself, including:
\begin{itemize}
  \item Number of nodes
  \item Number of nodes distinguished in different status
  \item Number of selections for pairs of nodes\footnote{may also include pair of ports for terminating grid network constructor} that scheduler had took
  \item Number of effective interactions the population executed
\end{itemize}

\par\noindent
Additionally, it provides a set of parameters' settings regarding the initial configuration of a protocol and a population to be simulated, which includes:
\begin{itemize}
  \item The number of nodes included in the simulation
  \item The initial state for each node\footnote{\noindent The state of edge for network constructors (i.e. network constructor and terminating grid network constructor) should be always "0" (i.e. inactivated) at initial, so it is omitted here.}
  \item The protocol type (and also different sets of transition rules for the protocol)
  \item Option on whether to use fast-forward simulation method for initially $n$-times selection, and the value of $n$ if the option is enabled.
  A fast-forward simulation executed in the model but does not present that process in the viewer so it normally faster than the case that does not enable this option.
\end{itemize}

\subsubsection{Evoluation of the project}

\par\noindent
In general, the simulator successfully implemented a series of different protocols for the three model mentioned above.

\paragraph{UI} The UI functions of simulator is verified through a large number of different population simulations. This ensures the UI functions work as they expected in design stage.
These experiments on theoretical model may also be asserted the correctness of model through the output configuration of these simulations.

\paragraph{Model} The model partition of the simulator developed through Testing driven development method. According to the specification, the testing suits had been written before the written of any
model code . The model code has to pass all testing suits after their implementation.
