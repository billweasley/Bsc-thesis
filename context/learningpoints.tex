\section{Learning Points}
This section will discussed some points that the author learned throughout the entire final year project process.
\subsection{Academic skills}
\paragraph{Basic understanding towards population protocols}
It was entirely an accidental for the author to select this topic as his final year project theme.
Dating back to one years' ago, the author saw the project description accidentally and find that it is possibly an attractive idea.
This is an area that few researchers focusing on compared with current research hotspot such as deep learning.
Though like that, the population protocol and network constructor provides a different perspective for what is computing and how computing works.

\paragraph{Applying mathematics into real project}
The research process involves some mathematics such as rotation matrix and trigonometric functions.
Though they are very basic, it is still a good practice to apply these knowledge learnt in book to solving the practical problems. It is truly
deepen the author's understanding towards linear algebra and trigonometric functions through working on this project.

\paragraph{Paper reading and information gathering}
Even for now, the author can only understand some partition of paper listed in bibliography but not all of them
because of his very limited knowledge, especially for some related mathematical knowledge, such as probability theory and
computational complexity theory.
However, this is also good train for the author to quick handle papers using his limited knowledge and to quick
gather useful information from related papers.

\paragraph{Academic writing and use of \LaTeX\ }
Though this is neither the first time that author write academic pieces nor first time
to use \LaTeX\ , this is still first time that the author using \LaTeX\ to finish a document that exceeds 20 pages.

\paragraph{Learning direction: Foundation knowledge}
The author found that he had some essential mathematic basics are not covered in his previous studies and
may start to his self-study in the near future.

\subsection{Technical (Development) skills}

\paragraph{Learnt a new language Kotlin and gained more understanding on programming language}
It was a proper choice for the author to take a risk for learning and using Kotlin rather than Java to finish
the project. Many language characteristics in Kotlin assisted the development process and simply the
implementation. The pattern matching, operator overloading and infix keyword deepen the author's understanding
towards programming language.

\paragraph{Had a better understanding on the relationship in between interface and model}
The application is all about how to cooperate the interface and model to enable them to work correctly and coherently.
Throughout the project, it took the author a large amount of time to construct the architecture of model and
user interface and to bridge them together.


\subsection{Soft skills}
\paragraph{Problem reduction and problem solving}
There were many problems that encountered in designed and development process finally
resolved through reducing problem to simpler case. For instance, the overlapping problem
for terminating grid network constructor could be resolved via discussion solution under
different situations. These solutions under different situations finally become the answer
to resolve the original problem.

\paragraph{Time management}
This entire development process teaches the author a lesson that practice
plan vitals. The author was failed to do so and led to a bad experience during
last period of development and dissertation writing.

\paragraph{Oral presentation}
The failure during requirement presentation enforced the author rethinking how to
properly deliver his context to his audience and optimise his strategy in his final
presentation through starting from easy concepts rather than presenting all core concepts at initial time of presentation.
